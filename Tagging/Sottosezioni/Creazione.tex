\subsection{Creazione di tag}
Prima di illustrare il comando per creare un tag è importante sapere che in Git esistono due tipi di tag:

\begin{itemize}
\item semplificati (lightweight);
\item commentati (annotated).
\end{center}

Un tag ''semplificato'' è semplicemente un riferimento ad uno specifico commit, mentre un tag ''commentato'' è un vero e proprio oggetto memorizzato nel database interno, ne viene calcolato il checksum e possiede un proprietario, proprio come un commit. Quest'ultimo tipo di tag può essere firmato e verificato con GNU privacyguard (GPG)

Per la creazione di tag ''commentati'' eseguire il comando:

\begin{center}
\texttt{git tag -a v1.0 -m ''Messaggio da allegare''}
\end{center}

l'opzione -a sta per ''annotated'' mentre ''v1.4'' è in nome del tag, l'ultimo parametro ''-m'' significa che verrà allegato come messaggio quello contenuto tra doppi apici. Se non viene specificato un messaggio verrà lanciato il tuo editor per inserirne uno.

Per avere informazioni su uno specifico tag, eseguire il seguente comando:

\begin{center}
\texttt{git show v1.0}
\end{center}

Per firmare un tag ''annotated'', tramite GPG e assumendo di possedere una chiave privata, lanciare il comando di creazione di un tag ''annotated'' passando come opzione ''$-$s'' al posto di $-$a.

Per creare un tag ''lightweight'' si usa il comando:

\begin{center}
\texttt{git tag v1.1}
\end{center}

ovvero lo stesso per i tag ''annotated'' senza nessuna opzione.