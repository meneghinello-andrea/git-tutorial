\subsection{Tramite shell dei comandi}
Per visualizzare la storia dei comandi tramite shell il comando da usare è il seguente:

\begin{center}
\texttt{git log}
\end{center}

questo comando è quello fondamentale che mostra tutta la storia del repository è possibile passare ulteriori parametri per affinara la ricerca come per esempio:

\begin{center}
\texttt{git log -n}
\end{center}

dove ''n'' va sostituito con un numero naturale, e Git ci restituirà gli utlimi ''n'' commit. Oppure per avere ogni commit su una singola riga:

\begin{center}
\texttt{git log --pretty=oneline}
\end{center}

Ma ve ne sono molti altri che è più utile imparare con la pratica che non con un elenco fine a se stesso.