\subsection{Creazione di un nuovo ramo}
Per la creazione di un nuovo ramo usiamo il comando

\begin{center}
\texttt{git branch NomeRamo}
\end{center}

dove ''NomeRamo'' andrà sostituito con un nome per noi significativo. Git per sapere a in quale ramo ci troviamo mantiene uno speciale puntatore, chiamato \textit{HEAD}.

E' \underline{importante} osservare che l'esecuzione del comando per creare un nuovo ramo non ci sposta da quello in cui attualmente siamo al nuovo ramo creato.
