\subsection{Alias}
Se con il proprio progetto ci si appoggia a server esterni, come github.com, per condividere il progetto con altri collaboratori è possibile definire un alias per evitare di ripetere l'inserimento dell'URL ad ogni invio/ricezione di modifiche.

Per definire un alias eseguire il seguente comando all'interno di uno specifico repository:

\begin{center}
\texttt{git remote add alias URL}
\end{center}

dove ''alias'' va sostituito con un nome significativo a piacere mentre ''URL'' va sostituito con l'indirizzo remoto dove risiede il repository.

Per rimuovere un alias è sufficiente il seguente comando all'interno del repository che contiene l'alias da eliminare:

\begin{center}
\texttt{git remote rm alias}
\end{center}

dove ''alias'' va sostituito con l'alias da rimuovere.

Per visualizzare gli alias configurati in un particolare repository digitare all'interno di esso, il comando:

\begin{center}
\texttt{git remote -v}
\end{center}

l'opzione ''-v'' specifica che oltre ai nomi vogliamo visionare l'URL.