\subsection{Configurazione iniziale}
Verrà ora illustrato come configurare alcune funzioni di Git secondo le esigenze personali del programmatore.

Lo strumento che permette di configuare Git si chiama ''git config'' e consiste in variabili che contengono i valori delle nostre configurazioni. Queste variabili possono trovarsi in:

\begin{itemize}
\item file ''/etc/gitconfig'' che contiene i valori per ogni utente del sistema e per tutti i repository configurati. Per leggere/scrivere questo file passiamo l'opzione ''--system'' a ''git config'';
\item file \textasciitilde /.gitconfig che contiene i valori per uno specifico utente del sistema e per i repository da lui configurati. Per leggere/scrivere questo file passiamo l'opzione ''--global'' a ''git config'';
\item file di configurazione locale che risiede nella cartella nascosta ''.git'' di ogni repository.
\end{itemize}

I valori di ogni livello sovrascrivono quelli di livello più alto, nello specifico se impostiamo un valore a livello di sistema e lo modifichiamo a livello locale di uno specifico repository, quello locale prevarrà su quello di sistema.

Su sistemi Windows il file ''.gitconfig'' si trova generalmente nella cartella:

\begin{center}
\texttt{C:\textbackslash Documents and Settings \textbackslash User}
\end{center}

per quanto riguarda il file ''/etc/gitconfig si trova all'interno della directory ''MSys'' e dipende da dove l'utente sceglie di installare il software.

\input{Configurazione/Sottosezioni/IdentitaPersona.tex}

\subsection{Editor di testo}
E' possibile personalizzare anche l'editor di testo utilizzato per inserire messaggi in Git. Generalmente Git utilizza Vi o Vim ma è possibile specificarne un altro con il comando:

\begin{center}
\texttt{git config --global core.editor gedit}
\end{center}

E' possibile sostituire ''gedit'' con il proprio editor predefinito. Per l'opzione ''--global'' vale quanto detto in precedenza.

\subsection{Diff personalizzato}
Lo strumento di ''diff'' viene usato da Git per risolvere i conflitti durante una fusione (merge). Per impostarne uno digitare il seguente comando:

\begin{center}
\texttt{git config --global merge.tool tool}
\end{center}

dove ''tool'' va sostituito con uno dei seguenti valori:

\begin{itemize}
\item vimdiff;
\item kdeff3;
\item tkdiff;
\item meld;
\item xxdiff;
\item emerge;
\item gvimdiff;
\item ecmerge;
\item opendiff.
\end{itemize}

Per l'opzione ''--global'' vale quanto detto in precedenza.

\subsection{Alias}
Se con il proprio progetto ci si appoggia a server esterni, come github.com, per condividere il progetto con altri collaboratori è possibile definire un alias per evitare di ripetere l'inserimento dell'URL ad ogni invio/ricezione di modifiche.

Per definire un alias eseguire il seguente comando all'interno di uno specifico repository:

\begin{center}
\texttt{git remote add alias URL}
\end{center}

dove ''alias'' va sostituito con un nome significativo a piacere mentre ''URL'' va sostituito con l'indirizzo remoto dove risiede il repository.

Per rimuovere un alias è sufficiente il seguente comando all'interno del repository che contiene l'alias da eliminare:

\begin{center}
\texttt{git remote rm alias}
\end{center}

dove ''alias'' va sostituito con l'alias da rimuovere.

Per visualizzare gli alias configurati in un particolare repository digitare all'interno di esso, il comando:

\begin{center}
\texttt{git remote -v}
\end{center}

l'opzione ''-v'' specifica che oltre ai nomi vogliamo visionare l'URL.

\subsection{Visualizzazione dei parametri configurati}
Per visualizzare i valori configurati digitare il seguente comando:

\begin{center}
\texttt{git config --list}
\end{center}

Invece se si vuole visualizzare un valore specifico digitare:

\begin{center}
\texttt{git config user.name}
\end{center}