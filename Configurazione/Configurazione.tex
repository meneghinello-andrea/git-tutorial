\subsection{Configurazione iniziale}
Verrà ora illustrato come configurare alcune funzioni di Git secondo le esigenze personali del programmatore.

Lo strumento che permette di configuare Git si chiama ''git config'' e consiste in variabili che contengono i valori delle nostre configurazioni. Queste variabili possono trovarsi in:

\begin{itemize}
\item file ''/etc/gitconfig'' che contiene i valori per ogni utente del sistema e per tutti i repository configurati. Per leggere/scrivere questo file passiamo l'opzione ''--system'' a ''git config'';
\item file \textasciitilde /.gitconfig che contiene i valori per uno specifico utente del sistema e per i repository da lui configurati. Per leggere/scrivere questo file passiamo l'opzione ''--global'' a ''git config'';
\item file di configurazione locale che risiede nella cartella nascosta ''.git'' di ogni repository.
\end{itemize}

I valori di ogni livello sovrascrivono quelli di livello più alto, nello specifico se impostiamo un valore a livello di sistema e lo modifichiamo a livello locale di uno specifico repository, quello locale prevarrà su quello di sistema.

Su sistemi Windows il file ''.gitconfig'' si trova generalmente nella cartella:

\begin{center}
\texttt{C:\textbackslash Documents and Settings \textbackslash User}
\end{center}

per quanto riguarda il file ''/etc/gitconfig si trova all'interno della directory ''MSys'' e dipende da dove l'utente sceglie di installare il software.

\input{Configurazione/Sottosezioni/IdentitaPersona.tex}