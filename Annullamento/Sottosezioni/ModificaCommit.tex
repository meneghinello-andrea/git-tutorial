\subsection{Modifica dell'ultimo commit}
Uno degli annullamenti più comuni è quando invii troppo presto un commit e magari dimentichi di aggiungere alcuni file, o dimentichi di inserire un messaggio. Per correggere questi tipi di errori si utilizza il comando:

\begin{center}
\texttt{git commit --amend}
\end{center}

Questo comando prende l'area di staging e la usa per il commit, quindi se abbiamo dimenticato alcuni file vanno aggiunti con il comando di ''add'' prima di lanciare questo. Se l'area di staging non è stata modificata allora possiamo cambiare il messaggio relativo al commit, infatti dopo aver digitato il comando si aprirà l'editor con il vecchio messaggio che avevamo inserito, lo modifichiamo e salviamo il tutto per ultimare la fase di commit.