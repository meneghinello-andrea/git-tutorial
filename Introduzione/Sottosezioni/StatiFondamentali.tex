\subsection{I tre stati fondamentali}
Git possiede tra stati fondamentali in cui possono trovarsi i file:

\begin{description}
\item[committed] ovvero il file è immagazzinato e al sicuro nel database locale;
\item[modified] ovvero il file è stato modificato ma non ancora inserito nel database locale;
\item[staged] ovvero che un file modificato è stato contrassegnato per essere inserito nello snapshot alla prossima operazione di commit;
\end{description}

di conseguenza l'ambiente è suddiviso anch'esso in tre zone:

\begin{itemize}
\item la directory di Git;
\item la directory di lavoro;
\item l'area di staging;
\end{itemize}

La directory di Git è il luogo dove sono memorizzati i metadati e il database degli oggetti. Questa è la parte fondamentale del repository ed è quella che viene copiata quando si esegue la clonazione del repository da un client ad un altro. Si tratta della cartella nascosta ''.git''.

La directory di lavoro è un singolo checkout di una versione del progetto. Sono file estratti dal database compresso, nella directory di Git, e posizionati nel disco per poter essere usati o modificati.

L'area di staging consiste in un file, che risiede nella cartella di Git, contenenete le informazioni per il successivo commit.

Da quanto appreso possiamo dire che il normale flusso di lavoro consiste in:

\begin{enumerate}
\item modifica di un file nella directory di lavoro;
\item esecuzione dell'operazione di staging per i file modificati;
\item esecuzione dell'operazione di commit per memorizzare in modo permanente i dati.
\end{enumerate}