\subsection{Un po di storia}

Inizialmente per tenere traccia delle versioni di un software i programmatori adottavano il cosiddetto Sistema di Controllo di Versione Locale che consisteva nel copiare, manualmente, una specifica versione in una nuova directory. Il sistema è molto semplice ma anche soggetto a molti errori. 

Per risolvere il problema, si svilupparono i Sistemi di Controllo di Versione (Version Control System, VCS), formati da un database che manteneva tutti cambiamenti dei file sotto controllo di versione.

Successivamente si dovette risolvere il problema che programmatori su uno stesso progetto dovevano collaborare tra loro, vennero cosi sviluppati i Sistemi di Controllo di Versione Distribuiti (Centralized Version Control System, CVCS). Questi sistemi hanno un unico server centrale che contiene tutte le versioni e gli utenti che possono amministrarli. Per alcuni anni questo fu lo standard maggiormente utilizzato.

Questo sistema presenta i seguenti vantaggi:

\begin{itemize}
\item chiunque sa, ad un certo punto dello sviluppo, cosa stanno facendo gli altri utenti;
\item gli amministratori hanno un controllo immediato e preciso su chi può fare cosa;
\item è più facile amministrare un database centrale che non uno su ogni client.
\end{itemize}

Ma presenta pure dei svantaggi:

\begin{itemize}
\item Il progetto è memorizzato in unico punto, quindi aumentano le possibilità di perdere tutto il lavoro, o di rallentarlo in caso di guasti momentanei al server.
\end{itemize}

Infine, memori dell'esperienza, si crearono i Sistemi di Controllo di Versione Distribuiti, (Distribuited Version Control System, DVCS) ovvero ogni client possiede una copia locale dell'intero repository. In questo modo se si blocca un server la copia di un qualsiasi client può essere usata per ripristinare la copia nel server.

Parliamo ora un pò più nel dettaglio di Git. Git è un DVCS scritto principalmente in C, esso nacque nel 2005 dopo la frattura nel rapporto tra la comunità che ha sviluppato il kernel Linux e la società che ha sviluppato BitKeeper. Questa frattura permise a Linus Torvalds di sviluppare un proprio strumento avendo come obbiettivi:

\begin{itemize}
\item velocita;
\item design semplice;
\item forte supporto allo sviluppo non-lineare (possibilità di migliaia di rami paralleli);
\item completamente distribuito;
\item capacità di gestire, in modo efficiente (velocità e dimensione dei dati), grandi progetti come il kernel Linux.
\end{itemize}

Una curiosità su questo sistema rigurda il nome associato al sistema. Infatti il nome dato da Linus Torvalds è un termine gergale britannico per indicare una persona stupida o sgradevole, infatti venne cosi motivato dallo stesso autore:

\begin{quotation}
''Sono un egoista bastardo, e do a tutti i miei progetti un nome che mi riguardi. Prima Linux, ora Git''.
\end{quotation}

Il wiki ufficiale tuttavia fornisce spiegazioni alternative. Per esempio, a causa della difficoltà di utilizzo delle prime versioni, il programma è stato definito:

\begin{quotation}
''Il sistema di controllo versione progettato per farti sentire più stupido di quanto tu non sia.''
\end{quotation}