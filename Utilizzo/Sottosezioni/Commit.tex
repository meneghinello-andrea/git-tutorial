\subsection{Eseguire il commit dei file in staging}
Dopo aver aggiunto i file nell'area di stage si possono versionare tramite l'operazione di commit.

Per eseguire il commit è sufficiente lanciare il comando:

\begin{center}
\texttt{git commit}
\end{center}

ottenendo come risultato la visualizzazione dell'editor predefinito per l'inserimento del messaggio di commit. Il file contiene come messaggio predefinito, commentato, l'ultimo output del comando di visualizzazione dello stato e la prima riga vuota. Nella prima riga si dovrà inserire il testo del messaggio. Si ricordi inoltre che lasciare la riga vuota farà fallire l'operazione di commit.

Si può comunque eseguire il tutto dalla riga di comando, attraverso il seguente comando:

\begin{center}
\texttt{git commit -m "Messaggio del commit"}
\end{center}

dove l'opzione ''-m'' indica che il testo racchiuso tra le virgilette sarà considerato come il testo dell'operazione di commit. Anche qui lasciare vuoto il campo farà fallire l'operazione.