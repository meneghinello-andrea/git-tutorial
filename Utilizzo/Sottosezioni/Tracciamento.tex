\subsection{Tracciamento dei file sorgente}
Dopo aver creato o modificato un file sorgente dobbiamo informare Git che vogliamo versionare tale/i file/s e questo avviene con il comando:

\begin{center}
\texttt{git add NomeFile}
\end{center}

dove con NomeFile si intende il percorso, dalla cartella principale del repository, del file da tracciare. Se NomeFile è una directory, il comando aggiungerà ricorsivamente tutti i file presenti in essa.

Se visualizziamo ora lo stato del repository notiamo che Git ci informa che alcuni file sono in stage. Lo si nota dal risulatato del comando di visualizzazione dello stato \textit{On branch master Changes to be committed}. Alla prossima operazione di commit il file verrà aggiunto alla versione registrata quando lo abbiamo tracciato con il comando ''git add''. Se nel frattempo avessimo modificato tale/i file/s dobbiamo ripetere il comando ''git add'' per tale/i file/s. Quest'ultima situazione è rappresentata dal fatto che se richiediamo lo stato ottenianmo la sezione \textit{Changed but not updated}.