\subsection{Movimenti di file}
A differenza di altri VCS, Git non traccia esplicitamente i movimenti di file. Se rinomini un file in Git nessun metadata è immagazzinato in Git che ti dirà che il file è stato rinominato. Per eseguire correttamente questa operazione dobbiamo utilizzare il comando:

\begin{center}
\texttt{git mv FileFrom FileTo}
\end{center}

dove con ''FileFrom'' e ''FileTo'' si intendo i percorsi di origine e destinazione del file, che può non solo essere spostato nell'albero ma anche rinominato. Il precedente comando è comodo perchè è come se ne eseguissimo tre, più precisamento sono:

\begin{center}
\texttt{mv FileFrom FileTo}\\
\texttt{git rm FileFrom}\\
\texttt{git add FileTo}
\end{center}