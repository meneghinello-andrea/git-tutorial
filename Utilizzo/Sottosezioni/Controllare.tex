\subsection{Controllare lo stato del repository}
Il comando da utilizzare per vedere lo stato del repository è:

\begin{center}
\texttt{git status}
\end{center}

Se come risultato si ottiene \textit{On branch master nothing to commit (working directory clean), significa che la cartella è pulita ossia Git non ha rilevato la presenza di nuovi file e/o file modificati. Altrimenti può essere visulizzato \textit{On branch master Untracked file} seguito dall'elenco dei file non tracciati, ossia non presenti nello snapshot precedente. I file non tracciati resteranno in questo stato finchè non diremo esplicitamente a Git di tracciarli, questo serve per impedire che vengano aggiunti in automatico file di compilazione o simili.